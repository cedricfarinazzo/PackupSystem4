\section{Présentation des membres du groupe}

    \subsection{MAUBANC Rémi}
        Elève standard dans cette école, j'ai été accepté dans ce groupe pour mes réalisations extra-scolaire qui est le développement et la mise en production d'un site d'aide aux révisions. Mais également pour des notes pas trop mauvaise en programmation en Sup et pour ma motivation. Ce projet me permettra d'être plus à l'aise avec le language C que je ne maîtrise que peu.
        
        
    \subsection{RIDEAU Cyril}
        Elève en seconde année à l'EPITA, je suis un grand fan de jeu vidéo, et de programmation. J'ai commencé la programmation en terminale car je souhaitais comprendre comment les programmes que j'appréciais tant fonctionnaient. J'ai donc rejoins l'EPITA. Ce projet m'intéresse car il touche à la gestion I/O\footnote{Input/Output en francais Entrée/Sortie ici du programme} et la gestion de fichier. J'aime beaucoup cette partie de la programmation car cela permet de faire beaucoup de choses complexes. J'espère donc que ce projet me permettra de me développer sur ce plan.
        
    \subsection{CASIER Frédéric}
    	Actuellement en seconde année à EPITA, je suis un passionné d’art et d’informatique, deux domaines qui occupent la majorité de mon temps libre. J’ai commencé à coder en classe de Terminale, en apprenant les bases du langage Python. Dans le domaine de l'informatique, je suis principalement intéressé par le développement d'interface graphique et le traitement d'image. Toutefois, si j'ai rejoint ce projet, ce n'est pas pour retravailler sur une petite interface, mais pour découvrir d'autres domaines et prendre de l'expérience sur une facette de la programmation où je n'excelle pas.
        
        
    \subsection{FARINAZZO Cédric}
        Passionné d’informatique, ce projet m’intéresse énormément. Le fonctionnement de ce projet me semble complet du point de vue algorithmique. Je suis à l’aise dans la plupart des langages de programmation. Je n’aurai pas trop de problème à réaliser ce projet. Ce projet sera donc pour moi un moyen de découvrir en détail le monde de la cryptographie.
        
\newpage

\section{Répartition des tâches}
    
    Voici la répartition des tâches pour la réalisation du projet : \\ \\

    {\normalsize
    	\begin{tabular}{|p{6cm}|p{1.6cm}|p{1.6cm}|p{1.6cm}|p{1.6cm}|}
    		\hline
    		Tâches & \multicolumn{4}{|c|}{Personnes} \\ 
    		\cline{2-5}
    			& Rémi & Cyril & Frédéric & Cédric \\
    		\hline
    		Système de fichier &  &  & X & X \\
    		\hline
    		Sauvegarde &  & X &  &  \\
    		\hline
    		Compression & X & X &  &  \\
    		\hline
    		Chiffrement &  & X &  & X \\
    		\hline
    		Site web & X &  &  & X \\
    		\hline
    		Interface graphique & X &  & X &  \\
    		\hline
    	\end{tabular}
    	\label{répartition}		
    }
